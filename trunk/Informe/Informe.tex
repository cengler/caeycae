\documentclass[a4paper, 10pt, notitlepage]{article}

\usepackage{moreverb} %para importar codigo

\usepackage{pepotina} %paquete personal para la caratula del DC

\usepackage[spanish,activeacute]{babel}
\usepackage{babel} %paquete de idioma

\usepackage[latin1]{inputenc}

\usepackage{color}

\usepackage{fancyhdr} %linea sup con comentarios

%\usepackage{listings}

\usepackage{lscape} %para hoja apaisada

\usepackage{framed} %para crear cajas de texto

\usepackage{lastpage} %ultima pagina

\addtolength{\topmargin}{-50pt} 
\addtolength{\textwidth}{105pt}
\addtolength{\textheight}{120pt}
\addtolength{\oddsidemargin}{-50pt}

\newcommand{\imagen}[3]
{
	\begin{figure}[htbp]
	  \centering
	    \includegraphics[scale=0.5]{#1}
	  \caption{#3}
	\end{figure}
}
%%% Encabezado y pie de p'agina
\pagestyle{fancy}
\fancyhead[LO]{Algoritmos y Estructura de Datos III}
\fancyhead[C]{}
\fancyhead[RO]{P\'agina \thepage\ de \pageref{LastPage}}
\renewcommand{\headrulewidth}{0.4pt}
\fancyfoot{}



\newcommand{\falta}{ \begin{framed}	\begin{center} \hspace{1cm} \Large FALTA \normalsize \hspace{1cm} \end{center} \end{framed}}

\begin{document}

\universidad{Universidad de Buenos Aires}
\facultad{Facultad de ciencias exactas y naturales}
\departamento{Departamento de Computacion}
\materia{Sistemas Operativos}
\resumen{En este trabajo se encontrar� los resultados obtenidos sobre una serie de ejercicios que pondran en practica los conocimientos adquiridos en la materia, Sistemas Operativos.\\ Los temas a tratar son los siguientes: Instalacion, utilizacion de los principales comandos, compilacion de programas y modificacion de politicas del sistema operativo}
\keys{MINIX, Sistemas Operativos, UNIX, KERNEL}
\titulo{Trabajo Practico de MINIX}
\subtitulo{Grupo N� XXX}
\fecha{13/10/2007 - 2do Cuatrimeste 2007}
\footspace{1cm}
%\integrante{Garc\'ia, Ana Daniela}{530/05}{danita719@yahoo.com.ar}
\integrante{Engler, Christian Alejandro}{caeycae@gmail.com}

%caratula
%\maketitle{}

%enunciado
%%%%%%%%%%%%%%%%%%%%%%%%%%%%%%%%%%%%%%%%%%%%%%%%%%%%%%%%%%%%%%%%%%%%%%%%%%%%%%%%%%%%%%%%%%%%
\section{Instalacion del Sistema Operativo}
%%%%%%%%%%%%%%%%%%%%%%%%%%%%%%%%%%%%%%%%%%%%%%%%%%%%%%%%%%%%%%%%%%%%%%%%%%%%%%%%%%%%%%%%%%%%
\subsection{A - Indique graficamente el layout del sistema operativo MINIX indicando sus componentes y las funciones que brindan. En especial los system\_calls que contienen.}

\vskip1cm

MINIX esta estructurado en 4 capas bien definidas.\footnote{Sistemas Operativos \textit{Dise�o e implementacion} Andrew S. Tanenbaum. CAP 2 SEC 2.5 pag 94}\\
Cada capa realiza una funcion bien definida y se comunica por medio de mensajes. Gracias a su modularidad el sistema operativo MINIX es facilmente modificable, pues no es necesario conocer la implementacion de las demas secciones.

\imagen{Layout.png}{}{El Layout de MINIX.}

\begin{itemize}
\item Capa 1\\
Es encarga del manejo de hardware, atrapando todas las interrupciones y se�ales que se lanzan, realiza la planificacion, cambios de contexto (salvando y restaurando registros) y ofece a las capas superiores una interface para el envio de mensajes, y las tares de bajo nivel correspondientes para que funcione bien la abstraccion de procesos.
\item Capa 2\\
Contiene las taras de Entrada/Salida para cada tipo de dispositivo. Se requiere una tarea particular para cada tipo de dispositivo, incuidos discos, impresoras, interfaces de red, relojes y cualquier otro dispositivo que desee comunicarse con el sistema. Es decir, contiene todo el conjunto de drivers necesarios para manipular todos los dispositivos del sistema. Junto a la capa 1, forman un solo programa binario llamado KERNEL, aunque su planificacion se hace de forma separada y se comunican utilizando mensajes.
\item Capa 3\\
Contiene todos los procesos que tienen el fin de brindar servicios utiles a los procesos de usuario. Como lo son la administracion de memoria y el sistema de archivos. Ejecutan a un nivel menos privilegiado que el kernel y ejecutan las llamadas al sistema operativo MINIX relativas a la administracion de memoria (MM) y sistema de archivos (FS). Este nivel se encarga en su mayoria en interceptar las llamadas al sistema operativo, mientras que el KERNEL, de la administracion de recursos.
\item Capa 4\\
En la capa superior se encuentran los procesos de usuarios: shells, editores compiladores y programas escritos por el usuario.
\end{itemize}

\newpage

\textbf{System Calls}\footnote{Sistemas Operativos \textit{Dise�o e implementacion} Andrew S. Tanenbaum. CAP 1 SEC 1.4 pag 23} de MINIX

\begin{itemize}

\item Llamadas al sistema para la Administraci�n de Procesos:

\begin{flushleft}
\begin{tabular}{ll}
pid = fork ()						& Crea un proceso hijo id�ntico al padre \\
pid = waitpid(pid, \&statloc, opts)	& Espera a que un hijo termine y toma su condici�n de\\
									& salida\\
s = wait (\&status)					& Versi�n vieja de waitpid\\
s = execve (name, argv, envp)		& Sustituye la imagen en memoria de un proceso\\
exit (status)						& Finaliza la ejecuci�n del proceso y devuelve su estado\\
size = brk (addr)					& Fija el tama�o del segmento de datos\\
pid = getpid ()						& Devuelve el id del proceso que lo invoca\\
pid = getpgrp ()					& Devuelve el id de grupo de procesos del invocador\\
pid = setsid ()						& Crea una nueva sesi�n y devuelve su id de grupo de\\
									& procesos\\
s = ptrace (req, pid, addr, data)	& Se usa para depurar\\
\end{tabular}
\end{flushleft}


\item Llamadas al sistema para Se�alizaci�n:

\begin{flushleft}
\begin{tabular}{ll}
s = sigaction (sig, \&act, \&oldact) & Define la acci�n a emprender al recibir se�ales.\\
s = sifreturn(\&context)			& Regresa una se�al.\\
s = sigprocmask(how,\&set,\&old)	& Examina o cambia la m�scara de se�al.\\
s = sigpending(set)					& Obtiene el conjunto de se�ales bloqueadas.\\
s = sigsuspend(sigmask)				& Sustituye la m�scara de se�al y suspende el proceso.\\
s = kill(pid,sig)					& Env�a una se�al a un proceso.\\
residual = alarm(seconds)			& Pone la alarma del reloj.\\
s = pause()							& Suspende el invocador hasta la pr�xima se�al.\\
\end{tabular}
\end{flushleft}


\item Llamadas al sistema para la Administraci�n de Archivos:

\begin{flushleft}
\begin{tabular}{ll}
fd = creat(name,mode)		& Forma obsoleta de crear un archivo nuevo.\\
fd = mknod(name,mode,addr)	& Crea un inodo normal, especial o de directorio.\\
fd = open(file,how,...)		& Abre un archivo para leer, escribir o ambas cosas.\\
s = close(fd) 				& Cierra un archivo abierto.\\
n = read(fd,buffer,nbytes)	& Lee datos de un archivo coloc�ndolos en un buffer.\\
n = write(fd,buffer,nbytes)	& Escribe datos de un buffer a un archivo.\\
pos = lseek(fd,offset,whence)	& Mueve el apuntador de archivos.\\
s = stat(name,\&buf)		& Obtiene la informaci�n de estado de un archivo.\\
s = fstat(fd,\&buf)			& Obtiene la informaci�n de estado de un archivo.\\
fd = dup(fd)				& Asigna un nuevo descriptor de archivo a un archivo abierto.\\
s = pipe(\&fd[0])			& Crea un conducto.\\
s = ioctl(fd, request, argp) & Realiza operaciones especiales con un archivo.\\
s = access(name,amode)		& Verifica la accesibilidad de un archivo.\\
s = rename(old,new)			& Da a un archivo un nuevo nombre.\\
s = fcntl(fd,cmd,...)		& Bloqueo de archivos y otras operaciones.\\
\end{tabular}
\end{flushleft}


\item Llamadas al sistema para la Administraci�n de Directorios y Sistema de Archivos:

\begin{flushleft}
\begin{tabular}{ll}
s = mkdir(name,mode)				& Crea un nuevo directorio.\\
s = rmdir(name)					& Elimina un directorio vac�o.\\
s = link(name1,name2)			& Crea una nueva entrada, name2, que apunta a name1\\
s = unlink(name)					& Elimina una entrada de directorio.\\
s = mount(special, name, flag)	& Monta un sistema de archivos.\\
s = umount(special)				& Desmonta un sistema de archivos.\\
s = sync()						& Desaloja todos los bloques en cach� al disco.\\
s = chdir(dirname)				& Cambia el directorio de trabajo.\\
s = chroot(dirname)				& Cambia el directorio ra�z.\\
\end{tabular}
\end{flushleft}


\item Llamadas al sistema para Protecci�n

\begin{flushleft}
\begin{tabular}{ll}
s = chmod(name, mode)	& Cambia los bits de protecci�n de un archivo.\\
uid = getuid()			& Obtiene el uid del invocador.\\
gid = getgid()			& Obtiene el gid del invocador.\\
s = setuid(uid)			& Establece el uid del invocador.\\
s = setgid(gid)			& Establece el gid del invocador.\\
s = chown(name, owner,group) & Cambia el propietario y el grupo de un archivo.\\
oldmask = umask(complmode)	& Cambia la m�scara de modo y devuelve la m�scara anterior.\\
\end{tabular}
\end{flushleft}


\item Llamadas al sistema para la Administraci�n del Tiempo

\begin{flushleft}
\begin{tabular}{ll}
seconds =time(\&seconds)	& Obtiene el tiempo transcurrido desde el 1ro de enero de 1970.\\
s =stime(tp)	& Establece el tiempo transcurrido desde el 1ro de enero de 1970.\\
s =utime(file,timep)	& Establece el tiempo de ``ultimo acceso'' de un archivo.\\
s =times(buffer)	& Obtiene el tiempo de usuario y sistema gastados hasta este\\ 						& momento.\\
\end{tabular}
\end{flushleft}

\end{itemize}

\subsection{B - Instalar MINIX en su version para DOS (DOSMINIX), WINDOWS o LINUX (BOCHS) segun se describe en\\ http://www.dc.uba.ar/people/materias/so/html/minix.html}

\vskip1cm

Instalamos MINIX sin particionar el rigido en WINDOWS, con el emulador \textbf{Qemu Manager 4.0} que no requiere instalacion y de esta manera nos facilitaba el trabajo a la hora de cambiar de maquina.

\begin{itemize}
	\item DOWNLOADS
	
		\begin{itemize}
			\item Qemu 4 + Qemu Manager, en la version que no requiere instalacion\\
			http://www.davereyn.co.uk/qem/qemumanager40.zip
			\item Disquette de instalacion de MINIX 144M.dsk\\
			http://www-2.dc.uba.ar/materias/so/minix/simulado/win/winminix2.zip
			\item Comandos extras\\
			http://www-2.dc.uba.ar/materias/so/minix/img-386/Base/USR.TAZ
			\item Fuentes del sistema\\
			http://www-2.dc.uba.ar/materias/so/minix/img-386/Source/SYS.TAZ
			\item Fuentes de comandos\\
			http://www-2.dc.uba.ar/materias/so/minix/img-386/Source/CMD.TAZ
			\item Programa para generar las imagenes de instalacion de MINIX.\\
			http://www-2.dc.uba.ar/materias/so/minix/img-386/PARTIR.EXE
		\end{itemize}
\end{itemize}


\falta
\newpage
%%%%%%%%%%%%%%%%%%%%%%%%%%%%%%%%%%%%%%%%%%%%%%%%%%%%%%%%%%%%%%%%%%%%%%%%%%%%%%%%%%%%%%%%%%%%
\section{Herramientas}
%%%%%%%%%%%%%%%%%%%%%%%%%%%%%%%%%%%%%%%%%%%%%%%%%%%%%%%%%%%%%%%%%%%%%%%%%%%%%%%%%%%%%%%%%%%%
\subsection{Indique que hace el comando make y mknode. Como se utilizan estos comandos en la instalacion de MINIX y en la creacion de un nuevo kernel. Para el caso del make muestre un archivo de ejemplo y explique que realiza cada uno de los comandos internos del archivo ejemplo.}

\vskip1cm

\falta
\newpage
%%%%%%%%%%%%%%%%%%%%%%%%%%%%%%%%%%%%%%%%%%%%%%%%%%%%%%%%%%%%%%%%%%%%%%%%%%%%%%%%%%%%%%%%%%%%
\section{Comandos basicos de MINIX/UNIX}
%%%%%%%%%%%%%%%%%%%%%%%%%%%%%%%%%%%%%%%%%%%%%%%%%%%%%%%%%%%%%%%%%%%%%%%%%%%%%%%%%%%%%%%%%%%%
\subsection{P�ngale password root al root.}

\vskip0.5cm

\begin{framed}
\begin{verbatim}
# passwd                                // comando para cambiar el password
Changing the shadow password of root
New password:                           // escribimos root
Retype password:                        // reescribimos root
#
# exit                                  // salimos de la sesion

Minix Release 2.0 Version 0

noname login: root                      // intentamos logeamos como root
Password:                               // escribimos algo != root
Login incorrect                         // la contrase�a no es correcta
login: root                             // nos volvemos a logar como root
Password:                               // escribimos root y entramos
#
\end{verbatim}
\end{framed}

\subsection{pwd}
Indique qu� directorio pasa a ser su current directory si ejecuta: 

\subsubsection{\# cd /usr/src}
\begin{framed}
\begin{verbatim}
# cd usr/src                            // cambia el current directory
# pwd                                   // nos muestra el current directory
/usr/src
#
\end{verbatim}
\end{framed}

\subsubsection{\# cd}
\begin{framed}
\begin{verbatim}
# cd                                    // cambia el current directory "al home"
# pwd                                   // nos muestra el current directory 
/
#
\end{verbatim}
\end{framed}

\subsubsection{�C�mo explica el punto 3.1.2?}
El comando cd sin parametros cambia el current directoy al \"home\" del usuario, como el usuario es root (administrador)
\falta 

\subsection{cat}
Cual es el contenido del archivo /usr/src/.profile y para que sirve.
\begin{framed}
\begin{verbatim}
# cat /usr/src/.profile                 // cat nos muestra el archivo
# Login shell profile.

# Environment.
umask 022
PATH=/usr/local/bin:/bin:/usr/bin 
PS1="! "
export PATH

# Erase character, erase line, and interrupt keys.
stty erase '^H' kill '^U' intr '^?'

# Check termianl type.
case $TERM in
dialup|unknown|network) 
        echo -n "Terminal type? ($TERM) "; read term"
        TERM="${term:-$TERM}"
esac

# Shell configuration.
case "$0" in *ash) . $HOME/.ashrc;; esac
#
\end{verbatim} 
\end{framed}

Cada usuario tiene asignado un shell o interprete de comandos. Nada impide que, en lugar de un
interprete de comandos, se asigne a un usuario un programa, cada vez que accediera al sistema, solo
podr�a ejecutar ese programa o aplicaci�n, y abandonar�a el sistema cada vez que finalizara dicho 
programa.
Una vez comprobada la validez del acceso y activado el shell, �ste lee el archivo /etc/profile, si existe, y
ejecuta todas las intrucciones que contenga. Entre las acciones principales que pueden indicarse en este 
archivo se pueden destacar:
\begin{itemize}
\item Visualizaci�n de los archivos de copyright y motd (mensajes que el administrador ponga para
informaci�n de los usuarios).
\item Establecimiento del TIMEZONE o huso horario. 
\item Indicaci�n de la existencia de correo.
\item Indicaci�n de la hora actual, y de la hora del �ltimo acceso.
\item Establecimiento de algunas variables de entorno (PATH, LOGNAME, etc.).
\end{itemize}
Una vez ejecutado el profile general, lee \$HOME/.profile con el cual se completar� la definici�n 
del entorno del usuario Entre las acciones que se pueden encontrar en este archivo destacan:
\begin{itemize}
\item Establecimiento de las variables de entorno definitivas.
\item Acciones personalizadas del arranque. 
\end{itemize}

\falta

revisar

\falta

\subsection{find}
En que directorio se encuentra el archivo \verb|proc.c|

\begin{framed}
\begin{verbatim} 
# find / -name proc.c -print            // buscamos el archivo
/usr/src/kernel/proc.c                  // tiene una unica aparicion en
#                                       // /usr/src/kernel/                               
\end{verbatim} 
\end{framed}

\subsection{mkdir}
Genere un directorio \verb|/usr/<nombregrupo>|
\begin{framed}
\begin{verbatim} 
# find / -name proc.c -print            // buscamos el archivo
# cd /usr                               // nos posicionamos en /usr
# pwd                                   // verificamos posicion
/usr
# mkdir grupo16                         // creamos la carpeta grupo16
# ls                                    // listamos archivos y directorios
adm  bin  grupo16  lib     man   preserve  src
ast  etc  include  local   mdec  spool     tmp      
# cd grupo16                            // ingresamos al directorio creado
#                       
\end{verbatim} 
\end{framed}

\subsection{cp}
Copie el archivo \verb|/etc/passwd| al directorio \verb|/usr/<nombregrupo>|
\begin{framed}
\begin{verbatim} 
# cp /etc/passwd /usr/grupo16           // copiamos el archivo
# cd /usr/grupo16                       // me posiciono en la cerpeta destino
# ls                                    // listo archivos y directorios
passwd                                  // 'passwd' se encuentra en el directorio
#
\end{verbatim} 
\end{framed}

\subsection{chgrp}
Cambie el grupo del archivo \verb|/usr/<grupo>/passwd| para que sea \verb|other|
\begin{framed}
\begin{verbatim} 
# cd /usr/grupo16                       // ingresamos al directorio
# ls -l                                 // listamos con -l que nos 
                                        // mas informacion del archivo
total 1                                 // existe solo un archivo
-rw-r--r--  1   root    operator    285  Oct  3  21:55  passwd
                                        // el grupo es: operator
# chgrp other passwd                    // cambiamos el grupo
# ls -l                                 // volvemos a listar
total 1                                 //
-rw-r--r--  1   root    other       285  Oct  3  21:55  passwd
                                        // y ahora el grupo es other
#
\end{verbatim} 
\end{framed}

\subsection{chown}
Cambie el propietario del archivo \verb|/usr/<grupo>/passwd| para que sea ast 
\begin{framed}
\begin{verbatim} 
# cd /usr/grupo16                       // ingresamos al directorio
# ls -l                                 // listamos
total 1                                 //
-rw-r--r--  1   root    other       285  Oct  3  21:55  passwd
                                        // el propietario es: root
# chown ast passwd                      // cambiamos el propietario
# ls -l                                 // volvemos a listar
total 1                                 //
-rw-r--r--  1   ast     other       285  Oct  3  21:55  passwd
                                        // y ahora el propietario es ast
#
\end{verbatim} 
\end{framed}



%\subsection{chmod
%Cambie los permisos del archivo /usr/<grupo>/passwd para que
%� el propietario tenga permisos de lectura, escritura y ejecuci�n
%� el grupo tenga solo permisos de lectura y ejecuci�n
%� el resto tenga solo permisos de ejecuci�n
%\subsection{grep
%Muestre las lineas que tiene el texto include en el archivo
%/usr/src/kernel/main.c
%Muestre las lineas que tiene el texto POSIX que se encuentren en todos los archivos 
%/usr/src/kernel/
%\subsection{su
%3.10.1. Para qu� sirve?
%3.10.2. Que sucede si ejecuta el comando su estando logueado como root?
%3.10.3. Genere una cuenta de <usuario>
%3.10.4. Entre a la cuenta <usuario> generada 
%3.10.5. Repita los comandos de 3.10.2
%\subsection{passwd
%3.11.1. Cambie la password del usuario nobody
%3.11.2. presione las teclas ALT-F2 y ver� otra sesion MINIX. Logearse como nobody
%3.11.3. ejecutar el comando su. 
%3.11.3.1. �Que le solicita ?
%3.11.3.2. �Sucede lo mismo que en 3.10.2? �Por qu�?
%\subsection{rm
%Suprima el archivo /usr/<grupo>/passwd
%\subsection{ln 
%Enlazar el archivo /etc/passwd a los siguientes archivos /tmp/contra1 /tmp/contra2
%Hacer un ls � l para ver cuantos enlaces tiene /etc/passwd
%\subsection{mkfs
%Genere un Filesystem MINIX en un diskette
%\subsection{mount 
%Montelo en el directorio
%/mnt
%Presente los filesystems que tiene montados
%\subsection{df
%Que espacio libre y ocupado tienen todos los filesystems montados? (En KBYTES)
%4
%\subsection{ps
%3.17.1. Cuantos procesos de usuario tiene ejecutando ? 
%3.17.2. Indique cuantos son del sistema
%\subsection{umount
%3.18.1. Desmonte el Filesystem del directorio
%/mnt
%3.18.2. Monte el Filesystem del diskette como read-only en el directorio
%/mnt
%3.18.3. Desmonte el Filesystem del directorio /mnt 
%\subsection{fsck
%Chequee la consistencia de Filesystem del diskette
%\subsection{dosdir
%Tome un diskette formateado en DOS con archivos y ejecute
%dosdir a
%Ejecute los comandos necesarios para que funcione correctamente el comando 
%anterior
%\subsection{dosread
%Copie un archivo de texto desde un diskette DOS
%al directorio /tmp
%\subsection{doswrite
%Copie el archivo /etc/passwd al diskette DOS


%\begin{framed}
%\begin{verbatim} 
%
%\end{verbatim} 
%\end{framed}
\newpage

%ejercicios
%\input{ejercicio1.tex}
%\newpage
%\input{ejercicio2.tex}
%\newpage
\tableofcontents
	
\end{document}

