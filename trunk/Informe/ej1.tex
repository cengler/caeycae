%%%%%%%%%%%%%%%%%%%%%%%%%%%%%%%%%%%%%%%%%%%%%%%%%%%%%%%%%%%%%%%%%%%%%%%%%%%%%%%%%%%%%%%%%%%%
\section{Instalacion del Sistema Operativo}
%%%%%%%%%%%%%%%%%%%%%%%%%%%%%%%%%%%%%%%%%%%%%%%%%%%%%%%%%%%%%%%%%%%%%%%%%%%%%%%%%%%%%%%%%%%%
\subsection{A - Indique graficamente el layout del sistema operativo MINIX indicando sus componentes y las funciones que brindan. En especial los system\_calls que contienen.}

\vskip1cm

MINIX esta estructurado en 4 capas bien definidas.\footnote{Sistemas Operativos \textit{Dise�o e implementacion} Andrew S. Tanenbaum. CAP 2 SEC 2.5 pag 94}\\
Cada capa realiza una funcion bien definida y se comunica por medio de mensajes. Gracias a su modularidad el sistema operativo MINIX es facilmente modificable, pues no es necesario conocer la implementacion de las demas secciones.

\imagen{Layout.png}{}{El Layout de MINIX.}

\begin{itemize}
\item Capa 1\\
Es encarga del manejo de hardware, atrapando todas las interrupciones y se�ales que se lanzan, realiza la planificacion, cambios de contexto (salvando y restaurando registros) y ofece a las capas superiores una interface para el envio de mensajes, y las tares de bajo nivel correspondientes para que funcione bien la abstraccion de procesos.
\item Capa 2\\
Contiene las taras de Entrada/Salida para cada tipo de dispositivo. Se requiere una tarea particular para cada tipo de dispositivo, incuidos discos, impresoras, interfaces de red, relojes y cualquier otro dispositivo que desee comunicarse con el sistema. Es decir, contiene todo el conjunto de drivers necesarios para manipular todos los dispositivos del sistema. Junto a la capa 1, forman un solo programa binario llamado KERNEL, aunque su planificacion se hace de forma separada y se comunican utilizando mensajes.
\item Capa 3\\
Contiene todos los procesos que tienen el fin de brindar servicios utiles a los procesos de usuario. Como lo son la administracion de memoria y el sistema de archivos. Ejecutan a un nivel menos privilegiado que el kernel y ejecutan las llamadas al sistema operativo MINIX relativas a la administracion de memoria (MM) y sistema de archivos (FS). Este nivel se encarga en su mayoria en interceptar las llamadas al sistema operativo, mientras que el KERNEL, de la administracion de recursos.
\item Capa 4\\
En la capa superior se encuentran los procesos de usuarios: shells, editores compiladores y programas escritos por el usuario.
\end{itemize}

\newpage

\textbf{System Calls}\footnote{Sistemas Operativos \textit{Dise�o e implementacion} Andrew S. Tanenbaum. CAP 1 SEC 1.4 pag 23} de MINIX

\begin{itemize}

\item Llamadas al sistema para la Administraci�n de Procesos:

\begin{flushleft}
\begin{tabular}{ll}
pid = fork ()						& Crea un proceso hijo id�ntico al padre \\
pid = waitpid(pid, \&statloc, opts)	& Espera a que un hijo termine y toma su condici�n de\\
									& salida\\
s = wait (\&status)					& Versi�n vieja de waitpid\\
s = execve (name, argv, envp)		& Sustituye la imagen en memoria de un proceso\\
exit (status)						& Finaliza la ejecuci�n del proceso y devuelve su estado\\
size = brk (addr)					& Fija el tama�o del segmento de datos\\
pid = getpid ()						& Devuelve el id del proceso que lo invoca\\
pid = getpgrp ()					& Devuelve el id de grupo de procesos del invocador\\
pid = setsid ()						& Crea una nueva sesi�n y devuelve su id de grupo de\\
									& procesos\\
s = ptrace (req, pid, addr, data)	& Se usa para depurar\\
\end{tabular}
\end{flushleft}


\item Llamadas al sistema para Se�alizaci�n:

\begin{flushleft}
\begin{tabular}{ll}
s = sigaction (sig, \&act, \&oldact) & Define la acci�n a emprender al recibir se�ales.\\
s = sifreturn(\&context)			& Regresa una se�al.\\
s = sigprocmask(how,\&set,\&old)	& Examina o cambia la m�scara de se�al.\\
s = sigpending(set)					& Obtiene el conjunto de se�ales bloqueadas.\\
s = sigsuspend(sigmask)				& Sustituye la m�scara de se�al y suspende el proceso.\\
s = kill(pid,sig)					& Env�a una se�al a un proceso.\\
residual = alarm(seconds)			& Pone la alarma del reloj.\\
s = pause()							& Suspende el invocador hasta la pr�xima se�al.\\
\end{tabular}
\end{flushleft}


\item Llamadas al sistema para la Administraci�n de Archivos:

\begin{flushleft}
\begin{tabular}{ll}
fd = creat(name,mode)		& Forma obsoleta de crear un archivo nuevo.\\
fd = mknod(name,mode,addr)	& Crea un inodo normal, especial o de directorio.\\
fd = open(file,how,...)		& Abre un archivo para leer, escribir o ambas cosas.\\
s = close(fd) 				& Cierra un archivo abierto.\\
n = read(fd,buffer,nbytes)	& Lee datos de un archivo coloc�ndolos en un buffer.\\
n = write(fd,buffer,nbytes)	& Escribe datos de un buffer a un archivo.\\
pos = lseek(fd,offset,whence)	& Mueve el apuntador de archivos.\\
s = stat(name,\&buf)		& Obtiene la informaci�n de estado de un archivo.\\
s = fstat(fd,\&buf)			& Obtiene la informaci�n de estado de un archivo.\\
fd = dup(fd)				& Asigna un nuevo descriptor de archivo a un archivo abierto.\\
s = pipe(\&fd[0])			& Crea un conducto.\\
s = ioctl(fd, request, argp) & Realiza operaciones especiales con un archivo.\\
s = access(name,amode)		& Verifica la accesibilidad de un archivo.\\
s = rename(old,new)			& Da a un archivo un nuevo nombre.\\
s = fcntl(fd,cmd,...)		& Bloqueo de archivos y otras operaciones.\\
\end{tabular}
\end{flushleft}


\item Llamadas al sistema para la Administraci�n de Directorios y Sistema de Archivos:

\begin{flushleft}
\begin{tabular}{ll}
s = mkdir(name,mode)				& Crea un nuevo directorio.\\
s = rmdir(name)					& Elimina un directorio vac�o.\\
s = link(name1,name2)			& Crea una nueva entrada, name2, que apunta a name1\\
s = unlink(name)					& Elimina una entrada de directorio.\\
s = mount(special, name, flag)	& Monta un sistema de archivos.\\
s = umount(special)				& Desmonta un sistema de archivos.\\
s = sync()						& Desaloja todos los bloques en cach� al disco.\\
s = chdir(dirname)				& Cambia el directorio de trabajo.\\
s = chroot(dirname)				& Cambia el directorio ra�z.\\
\end{tabular}
\end{flushleft}


\item Llamadas al sistema para Protecci�n

\begin{flushleft}
\begin{tabular}{ll}
s = chmod(name, mode)	& Cambia los bits de protecci�n de un archivo.\\
uid = getuid()			& Obtiene el uid del invocador.\\
gid = getgid()			& Obtiene el gid del invocador.\\
s = setuid(uid)			& Establece el uid del invocador.\\
s = setgid(gid)			& Establece el gid del invocador.\\
s = chown(name, owner,group) & Cambia el propietario y el grupo de un archivo.\\
oldmask = umask(complmode)	& Cambia la m�scara de modo y devuelve la m�scara anterior.\\
\end{tabular}
\end{flushleft}


\item Llamadas al sistema para la Administraci�n del Tiempo

\begin{flushleft}
\begin{tabular}{ll}
seconds =time(\&seconds)	& Obtiene el tiempo transcurrido desde el 1ro de enero de 1970.\\
s =stime(tp)	& Establece el tiempo transcurrido desde el 1ro de enero de 1970.\\
s =utime(file,timep)	& Establece el tiempo de ``ultimo acceso'' de un archivo.\\
s =times(buffer)	& Obtiene el tiempo de usuario y sistema gastados hasta este\\ 						& momento.\\
\end{tabular}
\end{flushleft}

\end{itemize}

\subsection{B - Instalar MINIX en su version para DOS (DOSMINIX), WINDOWS o LINUX (BOCHS) segun se describe en\\ http://www.dc.uba.ar/people/materias/so/html/minix.html}

\vskip1cm

Instalamos MINIX sin particionar el rigido en WINDOWS, con el emulador \textbf{Qemu Manager 4.0} que no requiere instalacion y de esta manera nos facilitaba el trabajo a la hora de cambiar de maquina.

\begin{itemize}
	\item DOWNLOADS
	
		\begin{itemize}
			\item Qemu 4 + Qemu Manager, en la version que no requiere instalacion\\
			http://www.davereyn.co.uk/qem/qemumanager40.zip
			\item Disquette de instalacion de MINIX 144M.dsk\\
			http://www-2.dc.uba.ar/materias/so/minix/simulado/win/winminix2.zip
			\item Comandos extras\\
			http://www-2.dc.uba.ar/materias/so/minix/img-386/Base/USR.TAZ
			\item Fuentes del sistema\\
			http://www-2.dc.uba.ar/materias/so/minix/img-386/Source/SYS.TAZ
			\item Fuentes de comandos\\
			http://www-2.dc.uba.ar/materias/so/minix/img-386/Source/CMD.TAZ
			\item Programa para generar las imagenes de instalacion de MINIX.\\
			http://www-2.dc.uba.ar/materias/so/minix/img-386/PARTIR.EXE
		\end{itemize}
\end{itemize}


\falta