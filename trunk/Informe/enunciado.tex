%%%%%%%%%%%%%%%%%%%%%%%%%%%%%%%%%%%%%%%%%%%%%%%%%%%%%%%%%%%%%%%%%%%%%%%%%%%%%%%%%%%%%%%%%%%%
\section{Introducci�n}
%%%%%%%%%%%%%%%%%%%%%%%%%%%%%%%%%%%%%%%%%%%%%%%%%%%%%%%%%%%%%%%%%%%%%%%%%%%%%%%%%%%%%%%%%%%%
\subsection{Enunciado}

\begin{framed}	

\begin{center}

\hspace{1cm}

\LARGE
UBA - FCEyN - Departamento de Computaci�n\\
\large
ALGORITMOS Y ESTRUCTURAS DE DATOS III\\
Trabajo Pr�ctico $N^o$ 1\\
Primera entrega: 7-5-2007, hasta las 19:30 horas\\
Segunda entrega: 28-5-2007, hasta las 19:30 horas\\
\small
Ver informaci�n general sobre los Trabajos Pr�cticos en la p�gina de la materia en Internet.
\end{center}

\hspace{1cm}

Ver informaci�n general sobre los Trabajos Pr�cticos en la p�gina de la materia en Internet.
El Booble art es un programa que muestra el mapa de Buenos Aires y sus atracciones tur�sticas. A medida que vamos recorriendo el mapa en la pantalla, podemos hacer un click en un punto y nos muestra varias fotos que corresponden a lugares que est�n cerca del punto que marcamos.

Nuestro objetivo es realizar un algoritmo que, dado un punto en el plano (la pantalla), decida cu�les son las fotos a mostrar.
Para esto, vamos a considerar los siguientes criterios:

\begin{itemize}
\item Cada foto tiene una posici�n en el plano dada por sus coordenadas.
\item La pantalla tiene un tama�o fijo de $s * t$
\item Al clickear en un punto del plano, las fotos que deben mostrarse son TODAS aquellas fotos que contienen a ese punto (es decir, el punto est� dentro del �rea dado por las coordenadas de la foto).
\item Las fotos pueden tener diferentes tama�os, intersecarse, estar contenidas una dentro de otra, etc.
\item Por supuesto, al listar las fotos a mostrar, pueden haber fotos que se intersecan, etc. Esto no nos preocupa, s�lo queremos conocer la lista de fotos a mostrarse en otra pantalla, donde otro programa (que se har� en Algoritmos 4) acomodar� estas fotos 
\end{itemize}

Vamos a realizar un preprocesamiento del conjunto de fotos para luego, cada vez que se realiza un click, podamos decidir qu� fotos deben mostrarse en forma eficiente.

\begin{enumerate}
\item (a) Dada un conjunto de fotos, construir un $arbol de intervalos$ usando \textbf{�rboles red black} (ser� explicado en clase) con las operaci�n $Insertar$.\vskip0.1cm
(b) Indicar el orden de las operaci�n $insertar$ No es obligatorio que figure en el informe la justificaci�n del orden de $insertar$ pero se preguntar� sobre esto durante el coloquio.\vskip0.1cm
(c) Calcular la complejidad \textsc{TEMPORAL Y ESPACIAL} de construir el �rbol de intervalos.\vskip0.1cm
(d) Dise�ar un algoritmo que, dado un punto en el plano, decida el conjunto de fotos a mostrar. Para esto, implementar la operaci�n $buscarInterseccion$.\vskip0.1cm
(e) Calcular la complejidad \textsc{TEMPORAL Y ESPACIAL} de realizar una consulta (llamamos ``consulta'' a clickear sobre un punto del plano).\vskip0.1cm
(f) Analizar el algoritmo midiendo el tiempo de ejecuci�n tanto para el armado del �rbol como para consultas. Es decir, para diferentes conjuntos de fotos (instancias), realizar varias consultas.\vskip0.1cm
\item Ahora, pensemos la siguiente variante. Cada tanto, el programa Booble art se actualiza y adiciona nuevas fotos a su mapa. Explicar c�mo cambia el algoritmo anterior, sus funciones, etc. Calcular la complejidad de la funci�n $insertar$ para este caso.
\end{enumerate}

Entrada \verb|Tp2Ej1.in| Este archivo contiene varias instancias de entrada. La primera l�nea contiene el n�mero de instancias. Cada instancia es definida de la siguiente manera. La primera l�nea contiene el n�mero de fotos $n$, las siguientes n l�neas contiene las coordenadas de las fotos: $x0$ $x1$ $y0$ $y1$, donde $x0 \leq x1$ y $y0 \leq  y1$.

\verb|Tp2Ej1d.in| Este archivo contiene consultas para hacer sobre todas las instancias del archivo anterior. La primera l�nea contiene la cantidad de consultas ($k$) y las siguientes $k$ l�neas contienen las coordenadas del punto del plano de la forma $x$ $y$.

NOTA: NO ES OBLIGATORIO QUE LA ENTRADA SE INGRESE POR LINEA DE COMANDOS, PERO EL QUE QUIERE PUEDE HACERLO DE AMBAS MANERAS.

Salida \verb|Tp2Ej4.out| Por cada instancia, cada l�nea contiene una foto que se debe mostrar identificada por sus cuatro coordenadas. La salida de cada instancias est� separada por un 0. 

Bibliograf�a: 

Fotocopiadora: ``Introduction to Algorithms'', Thomas H. Cormen, Charles E. Leiserson, Ronald L. Rivest, and Clifford Stein. Cap�tulo 13.

Sugerencia: Para buscar informaci�n sobre interval-trees pueden consultar en\\
\verb|http://www.dgp.toronto.edu/people/JamesStewart/378notes/22intervals/|

\end{framed}
\normalsize