%%%%%%%%%%%%%%%%%%%%%%%%%%%%%%%%%%%%%%%%%%%%%%%%%%%%%%%%%%%%%%%%%%%%%%%%%%%%%%%%%%%%%%%%%%%%
\section{Comandos basicos de MINIX/UNIX}
%%%%%%%%%%%%%%%%%%%%%%%%%%%%%%%%%%%%%%%%%%%%%%%%%%%%%%%%%%%%%%%%%%%%%%%%%%%%%%%%%%%%%%%%%%%%
\subsection{P�ngale password root al root.}

\vskip1cm

\begin{verbatim}
# passwd                                // comando para cambiar el password
Changing the shadow password of root
New password:                           // escribimos root
Retype password:                        // reescribimos root
#
# exit                                  // salimos de la sesion

Minix Release 2.0 Version 0

noname login: root                      // intentamos logeamos como root
Password:                               // escribimos algo != root
Login incorrect                         // la contrase�a no es correcta
login: root                             // nos volvemos a logar como root
Password:                               // escribimos root y entramos
#
\end{verbatim}

\subsection{\textbf{pwd}}

Indique qu� directorio pasa a ser su $current\ directory$ si ejecuta:

\subsubsection{$\sharp$ cd /usr/src}

%\begin{verbatim}
%    # cd /usr/src
%    # pwd
%    /usr/src
%\end{verbatim}

%Utilizando el comando \textbf{pwd} podemos ver en que directorio nos encontramos. En 'este caso el $current\ directory$ es \textbf{/usr/src}.

\subsubsection{$\sharp$ cd}

%\begin{verbatim}
%    # cd
%    # pwd
%    /
%    #
%\end{verbatim}

%Nuevamente utilizamos el comando \textbf{pwd} y el directorio actual es \textbf{/}.

\subsubsection{�C'omo explica el punto 1.3.2.2.?}

\falta