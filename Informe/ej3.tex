%%%%%%%%%%%%%%%%%%%%%%%%%%%%%%%%%%%%%%%%%%%%%%%%%%%%%%%%%%%%%%%%%%%%%%%%%%%%%%%%%%%%%%%%%%%%
\section{Comandos basicos de MINIX/UNIX}
%%%%%%%%%%%%%%%%%%%%%%%%%%%%%%%%%%%%%%%%%%%%%%%%%%%%%%%%%%%%%%%%%%%%%%%%%%%%%%%%%%%%%%%%%%%%
\subsection{P�ngale password root al root.}

\vskip0.5cm

\begin{framed}
\begin{verbatim}
# passwd                                // comando para cambiar el password
Changing the shadow password of root
New password:                           // escribimos root
Retype password:                        // reescribimos root
#
# exit                                  // salimos de la sesion

Minix Release 2.0 Version 0

noname login: root                      // intentamos logeamos como root
Password:                               // escribimos algo != root
Login incorrect                         // la contrase�a no es correcta
login: root                             // nos volvemos a logar como root
Password:                               // escribimos root y entramos
#
\end{verbatim}
\end{framed}

\subsection{pwd}
Indique qu� directorio pasa a ser su current directory si ejecuta: 

\subsubsection{\# cd /usr/src}
\begin{framed}
\begin{verbatim}
# cd usr/src                            // cambia el current directory
# pwd                                   // nos muestra el current directory
/usr/src
#
\end{verbatim}
\end{framed}

\subsubsection{\# cd}
\begin{framed}
\begin{verbatim}
# cd                                    // cambia el current directory "al home"
# pwd                                   // nos muestra el current directory 
/
#
\end{verbatim}
\end{framed}

\subsubsection{�C�mo explica el punto 3.1.2?}
El comando cd sin parametros cambia el current directoy al \"home\" del usuario, como el usuario es root (administrador)
\falta 

\subsection{cat}
Cual es el contenido del archivo /usr/src/.profile y para que sirve.
\begin{framed}
\begin{verbatim}
# cat /usr/src/.profile                 // cat nos muestra el archivo
# Login shell profile.

# Environment.
umask 022
PATH=/usr/local/bin:/bin:/usr/bin 
PS1="! "
export PATH

# Erase character, erase line, and interrupt keys.
stty erase '^H' kill '^U' intr '^?'

# Check termianl type.
case $TERM in
dialup|unknown|network) 
        echo -n "Terminal type? ($TERM) "; read term"
        TERM="${term:-$TERM}"
esac

# Shell configuration.
case "$0" in *ash) . $HOME/.ashrc;; esac
#
\end{verbatim} 
\end{framed}

Cada usuario tiene asignado un shell o interprete de comandos. Nada impide que, en lugar de un
interprete de comandos, se asigne a un usuario un programa, cada vez que accediera al sistema, solo
podr�a ejecutar ese programa o aplicaci�n, y abandonar�a el sistema cada vez que finalizara dicho 
programa.
Una vez comprobada la validez del acceso y activado el shell, �ste lee el archivo /etc/profile, si existe, y
ejecuta todas las intrucciones que contenga. Entre las acciones principales que pueden indicarse en este 
archivo se pueden destacar:
\begin{itemize}
\item Visualizaci�n de los archivos de copyright y motd (mensajes que el administrador ponga para
informaci�n de los usuarios).
\item Establecimiento del TIMEZONE o huso horario. 
\item Indicaci�n de la existencia de correo.
\item Indicaci�n de la hora actual, y de la hora del �ltimo acceso.
\item Establecimiento de algunas variables de entorno (PATH, LOGNAME, etc.).
\end{itemize}
Una vez ejecutado el profile general, lee \$HOME/.profile con el cual se completar� la definici�n 
del entorno del usuario Entre las acciones que se pueden encontrar en este archivo destacan:
\begin{itemize}
\item Establecimiento de las variables de entorno definitivas.
\item Acciones personalizadas del arranque. 
\end{itemize}

\falta

revisar

\falta

\subsection{find}
En que directorio se encuentra el archivo \verb|proc.c|

\begin{framed}
\begin{verbatim} 
# find / -name proc.c -print            // buscamos el archivo
/usr/src/kernel/proc.c                  // tiene una unica aparicion en
#                                       // /usr/src/kernel/                               
\end{verbatim} 
\end{framed}

\subsection{mkdir}
Genere un directorio \verb|/usr/<nombregrupo>|
\begin{framed}
\begin{verbatim} 
# find / -name proc.c -print            // buscamos el archivo
# cd /usr                               // nos posicionamos en /usr
# pwd                                   // verificamos posicion
/usr
# mkdir grupo16                         // creamos la carpeta grupo16
# ls                                    // listamos archivos y directorios
adm  bin  grupo16  lib     man   preserve  src
ast  etc  include  local   mdec  spool     tmp      
# cd grupo16                            // ingresamos al directorio creado
#                       
\end{verbatim} 
\end{framed}

\subsection{cp}
Copie el archivo \verb|/etc/passwd| al directorio \verb|/usr/<nombregrupo>|
\begin{framed}
\begin{verbatim} 
# cp /etc/passwd /usr/grupo16           // copiamos el archivo
# cd /usr/grupo16                       // me posiciono en la cerpeta destino
# ls                                    // listo archivos y directorios
passwd                                  // 'passwd' se encuentra en el directorio
#
\end{verbatim} 
\end{framed}

\subsection{chgrp}
Cambie el grupo del archivo \verb|/usr/<grupo>/passwd| para que sea \verb|other|
\begin{framed}
\begin{verbatim} 
# cd /usr/grupo16                       // ingresamos al directorio
# ls -l                                 // listamos con -l que nos 
                                        // mas informacion del archivo
total 1                                 // existe solo un archivo
-rw-r--r--  1   root    operator    285  Oct  3  21:55  passwd
                                        // el grupo es: operator
# chgrp other passwd                    // cambiamos el grupo
# ls -l                                 // volvemos a listar
total 1                                 //
-rw-r--r--  1   root    other       285  Oct  3  21:55  passwd
                                        // y ahora el grupo es other
#
\end{verbatim} 
\end{framed}

\subsection{chown}
Cambie el propietario del archivo \verb|/usr/<grupo>/passwd| para que sea ast 
\begin{framed}
\begin{verbatim} 
# cd /usr/grupo16                       // ingresamos al directorio
# ls -l                                 // listamos
total 1                                 //
-rw-r--r--  1   root    other       285  Oct  3  21:55  passwd
                                        // el propietario es: root
# chown ast passwd                      // cambiamos el propietario
# ls -l                                 // volvemos a listar
total 1                                 //
-rw-r--r--  1   ast     other       285  Oct  3  21:55  passwd
                                        // y ahora el propietario es ast
#
\end{verbatim} 
\end{framed}



%\subsection{chmod
%Cambie los permisos del archivo /usr/<grupo>/passwd para que
%� el propietario tenga permisos de lectura, escritura y ejecuci�n
%� el grupo tenga solo permisos de lectura y ejecuci�n
%� el resto tenga solo permisos de ejecuci�n
%\subsection{grep
%Muestre las lineas que tiene el texto include en el archivo
%/usr/src/kernel/main.c
%Muestre las lineas que tiene el texto POSIX que se encuentren en todos los archivos 
%/usr/src/kernel/
%\subsection{su
%3.10.1. Para qu� sirve?
%3.10.2. Que sucede si ejecuta el comando su estando logueado como root?
%3.10.3. Genere una cuenta de <usuario>
%3.10.4. Entre a la cuenta <usuario> generada 
%3.10.5. Repita los comandos de 3.10.2
%\subsection{passwd
%3.11.1. Cambie la password del usuario nobody
%3.11.2. presione las teclas ALT-F2 y ver� otra sesion MINIX. Logearse como nobody
%3.11.3. ejecutar el comando su. 
%3.11.3.1. �Que le solicita ?
%3.11.3.2. �Sucede lo mismo que en 3.10.2? �Por qu�?
%\subsection{rm
%Suprima el archivo /usr/<grupo>/passwd
%\subsection{ln 
%Enlazar el archivo /etc/passwd a los siguientes archivos /tmp/contra1 /tmp/contra2
%Hacer un ls � l para ver cuantos enlaces tiene /etc/passwd
%\subsection{mkfs
%Genere un Filesystem MINIX en un diskette
%\subsection{mount 
%Montelo en el directorio
%/mnt
%Presente los filesystems que tiene montados
%\subsection{df
%Que espacio libre y ocupado tienen todos los filesystems montados? (En KBYTES)
%4
%\subsection{ps
%3.17.1. Cuantos procesos de usuario tiene ejecutando ? 
%3.17.2. Indique cuantos son del sistema
%\subsection{umount
%3.18.1. Desmonte el Filesystem del directorio
%/mnt
%3.18.2. Monte el Filesystem del diskette como read-only en el directorio
%/mnt
%3.18.3. Desmonte el Filesystem del directorio /mnt 
%\subsection{fsck
%Chequee la consistencia de Filesystem del diskette
%\subsection{dosdir
%Tome un diskette formateado en DOS con archivos y ejecute
%dosdir a
%Ejecute los comandos necesarios para que funcione correctamente el comando 
%anterior
%\subsection{dosread
%Copie un archivo de texto desde un diskette DOS
%al directorio /tmp
%\subsection{doswrite
%Copie el archivo /etc/passwd al diskette DOS


%\begin{framed}
%\begin{verbatim} 
%
%\end{verbatim} 
%\end{framed}